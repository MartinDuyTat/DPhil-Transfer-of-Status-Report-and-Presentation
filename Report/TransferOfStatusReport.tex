%%%%%%%%%%%%%%%%%%%%%%%%%%%%%%%%%%%%%%%%%%%%%%%%%%%%%%%%%%%%%%%%%
\documentclass[12pt, a4paper, notitlepage, onecolumn]{article}
\usepackage[UKenglish]{babel}                   % UK style
\usepackage[utf8]{inputenc}
\usepackage[margin=1in]{geometry}               % Margin size
\usepackage{hyperref}                           % Coloured hyperlinks
  \hypersetup{colorlinks = true}
\usepackage{lmodern}                            % Modern fonts
\usepackage{graphicx}            % For figures
\usepackage[percent]{overpic}% For figures with text overlay
\usepackage{amsmath,amssymb}   % Mathematical symbols
\usepackage{mathtools}
\usepackage{siunitx}        % SI-units
\sisetup{exponent-product = \cdot}      % Dot product instead of cross product
\sisetup{separate-uncertainty = true}   % Plus-minus uncertainty
\usepackage{physics}                            % Elegant equations in physics
\usepackage{booktabs}                     % Nice lines, for instance in tables
\usepackage[font=small,labelfont=bf]{caption}% Caption
\usepackage{float}                              % Table do not move with [H].
\usepackage{subcaption}                         % For subfigures
\usepackage[en-GB]{datetime2}                          % UK date format
\usepackage{listings}                           %Source code
\usepackage{feynmp}                             % Feynman diagrams
\DeclareGraphicsRule{*}{mps}{*}{}               % Include Feynman diagrams
%%%%%%%%%%%%%%%%%%%%%%%%%%%%%%%%%%%%%%%%%%%%%%%%%%%%%%%%%%%%%%%
\title{Determination of the CKM angle $\gamma$ in $B^\pm\to DK^\pm, D\pi^\pm$ decays and strong phase determination of $D\to K^+K^-\pi^+\pi^-$ at BESIII}
\author{Martin Duy Tat}
\date{\today}
\numberwithin{equation}{section}
%%%%%%%%%%%%%%%%%%%%%%%%%%%%%%%%%%%%%%%%%%%%%%%%%%%%%%%%%%%%%%%
\begin{document}
\maketitle
\begin{abstract}
\noindent Write abstract at the end
\end{abstract}
%%%%%%%%%%%%%%%%%%%%%%%%%%%%%%%%%%%%%%%%%%%%%%%%%%%%%%%%%%%%%%%
\section{Introduction}
\noindent Introduce $\gamma$, GGSZ method and strong phase input from BESIII

\begin{figure}[H]
  \centering
  \vspace{0.3cm}
  \begin{subfigure}{0.5\textwidth}
    \centering
    \begin{fmffile}{fgraph_example1}
      \setlength{\unitlength}{0.4cm}
      \begin{fmfgraph*}(8,5)
        \fmfleft{i1,i2}
        \fmfright{o1,o2}
        \fmflabel{$e^-$}{i1}
        \fmflabel{$e^+$}{i2}
        \fmflabel{$e^-$}{o1}
        \fmflabel{$e^+$}{o2}
        \fmf{fermion}{i1,w1}
        \fmf{fermion}{w1,o1}
        \fmf{boson,label=$\gamma$}{w1,w2}
        \fmf{fermion}{w2,i2}
        \fmf{fermion}{o2,w2}
      \end{fmfgraph*}
    \end{fmffile}
  \end{subfigure}%
  \begin{subfigure}{0.5\textwidth}
    \centering
    \begin{fmffile}{fgraph_example2}
      \setlength{\unitlength}{0.4cm}
      \begin{fmfgraph*}(8,5)
        \fmfleft{i1,i2}
        \fmfright{o1,o2}
        \fmflabel{$e^-$}{i1}
        \fmflabel{$e^+$}{i2}
        \fmflabel{$e^-$}{o1}
        \fmflabel{$e^+$}{o2}
        \fmf{fermion}{i1,w1}
        \fmf{fermion}{w1,i2}
        \fmf{boson,label=$\gamma$}{w1,w2}
        \fmf{fermion}{w2,o1}
        \fmf{fermion}{o2,w2}
      \end{fmfgraph*}
    \end{fmffile}
  \end{subfigure}
  \vspace{0.3cm}
  \caption{Examples of Feynman diagrams for scattering of an electron $e^-$ and a position $e^+$.}
  \label{fig_feynman_diagram}
\end{figure}

%%%%%%%%%%%%%%%%%%%%%%%%%%%%%%%%%%%%%%%%%%%%%%%%%%%%%%%%%%%%%%%
\section{LHCb detector}
\noindent Briefly describe the VELO and RICH

\begin{figure}[H] 
  \centering
  \begin{subfigure}{0.5\textwidth}
    \centering
    \includegraphics[width=1\textwidth]{example-image-a}
    \caption{LHCb detector overview}
  \end{subfigure}%
  \begin{subfigure}{0.5\textwidth}
    \centering
    \includegraphics[width=1\textwidth]{example-image-a}
    \caption{LHCb detector overview}
  \end{subfigure}
  \caption{}
  \label{fig_lhcb_detector}
\end{figure}

%%%%%%%%%%%%%%%%%%%%%%%%%%%%%%%%%%%%%%%%%%%%%%%%%%%%%%%%%%%%%%%
\section{Binning scheme}
\noindent Describe the binning scheme developed and toy studies for testing it

%%%%%%%%%%%%%%%%%%%%%%%%%%%%%%%%%%%%%%%%%%%%%%%%%%%%%%%%%%%%%%%
\section{$B^\pm$ candidate selection}
\subsection{Signal candidate requirements}
\noindent Explain how signal events are selected

\begin{table}[H]
  \centering
  \caption{Requirements}
  \label{data}
  \begin{tabular}{cc} 
    \toprule
    Letter & Numerical value \\
    \midrule
    $c$ & $\SI{299792458}{\meter\per\second}$ \\
    $G$ & $\SI{6.67384e-11}{\newton\meter\squared\per\kilogram\squared}$ \\
    $\hbar$ & $\SI{1.05457e-34}{\joule\second}$ \\
    $k_B$ & $\SI{1.38065e-23}{\joule\per\kelvin}$ \\
    $e$ & $\SI{1.60218e-19}{\coulomb}$ \\
    \bottomrule
  \end{tabular}
\end{table}

\subsection{Background from $D^0\to K^-\pi^+\pi^-\pi^+$}
\noindent Show studies of $K3\pi$ contamination

\subsection{Charmless backgrounds}
\noindent Show how flight significance cut removes $B\to KKK\pi\pi$ and mention that $B\to KK\pi\pi\pi$ is insignificant

%%%%%%%%%%%%%%%%%%%%%%%%%%%%%%%%%%%%%%%%%%%%%%%%%%%%%%%%%%%%%%%
\section{Fit to extract CP observables}
\subsection{Global fit and invariant mass spectra}
\noindent State the fit procedure for global fit and show results of global fit for Run $2$, including yields

\subsection{Binned CP fit and CP observables}
\noindent Explain the binned CP fit to extract CP observables

\subsection{Validation of fit procedure with toy studies}

%%%%%%%%%%%%%%%%%%%%%%%%%%%%%%%%%%%%%%%%%%%%%%%%%%%%%%%%%%%%%%%
\section{Discussion of future work}
\noindent Discuss the plan further


%%%%%%%%%%%%%%%%%%%%%%%%%%%%%%%%%%%%%%%%%%%%%%%%%%%%%%%%%%%%%%%                                                                          
\bibliography{references}
\bibliographystyle{unsrt}

\end{document}