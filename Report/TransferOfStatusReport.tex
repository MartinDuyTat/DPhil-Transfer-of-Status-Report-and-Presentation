%%%%%%%%%%%%%%%%%%%%%%%%%%%%%%%%%%%%%%%%%%%%%%%%%%%%%%%%%%%%%%%%%
\documentclass[12pt, a4paper, notitlepage, onecolumn]{article}
\usepackage[UKenglish]{babel}                   % UK style
\usepackage[utf8]{inputenc}
\usepackage[margin=1in]{geometry}               % Margin size
\usepackage{hyperref}                           % Coloured hyperlinks
  \hypersetup{colorlinks = true}
\usepackage{lmodern}                            % Modern fonts
\usepackage{graphicx}            % For figures
\usepackage[percent]{overpic}% For figures with text overlay
\usepackage{amsmath,amssymb}   % Mathematical symbols
\usepackage{mathtools}
\usepackage{siunitx}        % SI-units
\sisetup{exponent-product = \cdot}      % Dot product instead of cross product
\sisetup{separate-uncertainty = true}   % Plus-minus uncertainty
\usepackage{physics}                            % Elegant equations in physics
\usepackage{booktabs}                     % Nice lines, for instance in tables
\usepackage[font=small,labelfont=bf]{caption}% Caption
\usepackage{float}                              % Table do not move with [H].
\usepackage{subcaption}                         % For subfigures
\usepackage[en-GB]{datetime2}                          % UK date format
\usepackage{listings}                           %Source code
\usepackage{feynmp}                             % Feynman diagrams
\DeclareGraphicsRule{*}{mps}{*}{}               % Include Feynman diagrams
\usepackage{scalerel}
\newcommand{\mylbrace}[2]{\vspace{#2pt}\hspace{6pt}\scaleleftright[\dimexpr5pt+#1\dimexpr0.06pt]{\lbrace}{\rule[\dimexpr2pt-#1\dimexpr0.5pt]{-4pt}{#1pt}}{.}}
\newcommand{\myrbrace}[2]{\vspace{#2pt}\scaleleftright[\dimexpr5pt+#1\dimexpr0.06pt]{.}{\rule[\dimexpr2pt-#1\dimexpr0.5pt]{-4pt}{#1pt}}{\rbrace}\hspace{6pt}}
%%%%%%%%%%%%%%%%%%%%%%%%%%%%%%%%%%%%%%%%%%%%%%%%%%%%%%%%%%%%%%%
\title{Determination of the CKM angle $\gamma$ in $B^\pm\to DK^\pm, D\pi^\pm$ decays and strong phase determination of $D\to K^+K^-\pi^+\pi^-$ at BESIII}
\author{Martin Duy Tat}
\date{\today}
\numberwithin{equation}{section}
%%%%%%%%%%%%%%%%%%%%%%%%%%%%%%%%%%%%%%%%%%%%%%%%%%%%%%%%%%%%%%%
\begin{document}
\maketitle
\begin{abstract}
\noindent Write abstract at the end
\end{abstract}
%%%%%%%%%%%%%%%%%%%%%%%%%%%%%%%%%%%%%%%%%%%%%%%%%%%%%%%%%%%%%%%
\section{Introduction}
\noindent In the Standard Model, CP-violation can occur if the CKM matrix has a non-trivial weak phase. This is studied by measuring the lengths and angles of the Unitary Triangle of the CKM matrix. In particular, the angle $\gamma = \arg(-V_{ud}V^*{ub}/V_{cd}V^*{cb})$ is the only angle that can be measured at tree level, with negligible theoretical uncertainties. A precise determination of $\gamma$ is therefore a good Standard Model benchmark which can be compared with indirect determinations from other CKM observables that are sensitive to new physics.

Sensitivity to $\gamma$ can be achieved through interference between the $b\to c\bar{u}s$ and $b\to u\bar{c}s$ transitions. A powerful decay mode is $B^\pm\to DK^\pm$, where $D$, a superposition of $D^0$ and $\bar{D^0}$, subsequently decays to a self-conjugate state. This is illustrated in Fig. \ref{fig_feynman_B2DK}. On the left, the colour favoured decay $B^-\to D^0K^-$ is shown, while on the right is the decay colour suppressed $B^-\to\bar{D^0}K^-$. Interference is observed when $D^0$ and $\bar{D^0}$ decays to a common final state.

\begin{figure}[H]
  \centering
  \vspace{0.3cm}
  \begin{subfigure}{0.5\textwidth}
    \centering
    \begin{fmffile}{fgraph/fgraph_BtoDK1}
      \setlength{\unitlength}{0.4cm}
      \begin{fmfgraph*}(6,6)
        \fmfstraight
        \fmfleft{i1,B,i2,t1,t2,t3,t9,t10}
        \fmfright{o1,D,o2,t4,t5,o3,K,o4}
        \fmflabel{$\bar{u}$}{i1}
        \fmflabel{$b$}{i2}
        \fmfv{l.d=20,l.a=180,l={$B^-$\mylbrace{30}{-8}}}{B}
        \fmflabel{$\bar{u}$}{o1}
        \fmflabel{$c$}{o2}
        \fmflabel{$\bar{u}$}{o3}
        \fmflabel{$s$}{o4}
        \fmfv{l.d=15,l.a=0,l={\myrbrace{30}{-12}}$D^0$}{D}
        \fmfv{l.d=15,l.a=0,l={\myrbrace{30}{11}}$K^-$}{K}
        \fmf{fermion}{o1,i1}
        \fmf{fermion,tension=1.5}{i2,v1}
        \fmf{fermion}{v1,o2}
        \fmf{phantom,tension=1.5}{t9,v2}
        \fmf{boson,label=$W$,label.side=left,tension=0}{v1,v2}
        \fmf{fermion}{v2,o4}
        \fmf{fermion}{o3,v2}
      \end{fmfgraph*}
    \end{fmffile}
    \vspace{0.5cm}
    \caption{$B^-\to D^0K^-$}
  \end{subfigure}%
  \begin{subfigure}{0.5\textwidth}
    \centering
    \begin{fmffile}{fgraph/fgraph_BtoDK2}
      \setlength{\unitlength}{0.4cm}
      \begin{fmfgraph*}(6,6)
        \fmfstraight
        \fmfleft{i1,t1,t2,B,t9,t10,i2}
        \fmfright{o1,K,o2,t4,t5,o3,D,o4}
        \fmflabel{$\bar{u}$}{i1}
        \fmflabel{$b$}{i2}
        \fmfv{l.d=20,l.a=180,l={$B^-$\mylbrace{100}{-8}}}{B}
        \fmflabel{$\bar{u}$}{o1}
        \fmflabel{$s$}{o2}
        \fmflabel{$\bar{c}$}{o3}
        \fmflabel{$u$}{o4}
        \fmfv{l.d=15,l.a=0,l={\myrbrace{30}{13}}$\bar{D^0}$}{D}
        \fmfv{l.d=15,l.a=0,l={\myrbrace{30}{-13}}$K^-$}{K}
        \fmf{fermion}{o1,i1}
        \fmf{fermion,tension=1.5}{i2,v1}
        \fmf{fermion}{v1,o4}
        \fmf{phantom,tension=1.5}{t2,v2}
        \fmf{boson,label=$W$,label.side=left,tension=0}{v1,v2}
        \fmf{fermion}{v2,o2}
        \fmf{fermion}{o3,v2}
      \end{fmfgraph*}
    \end{fmffile}
    \vspace{0.5cm}
    \caption{$B^-\to\bar{D^0}K^-$}
  \end{subfigure}
  \caption{Feynman diagrams of $B^-\to DK^-$ decays}
  \label{fig_feynman_B2DK}
\end{figure}

A wide range of subsequent $D$ decays has been studied. Most recently, the measurement $\gamma = (68.7^{+5.2}_{-5.1})^\circ$ from an analysis of the decay modes $D\to K_S^0\pi^+\pi^-$ and $D\to K_S^0K^+K^-$ was obtained, which is the single most precise measurement of $\gamma$. In this project, the decay $B^\pm\to DK^\pm$, where $D\to K^+K^-\pi^+\pi^-$, is considered. An initial study showed that a precision of $\SI{14}{\degree}$ is achievable with a sample of $1000$ $B^\pm\to DK^\pm$ candidates. From similar decay channels, it is estimated that $2000$ candidates can be reconstructed from the combined Run $1$+$2$ LHCb dataset.

A significant challenge with this analysis is that the $D\to K^+K^-\pi^+\pi^-$ decay is a multi-body decay, so the strong phase difference between the $D^0$ and $\bar{D^0}$ decays varies non-trivially across phase space. Moreover, with four particles in the final state phase space becomes five-dimensional. To predict this strong phase difference, a decay model, such as one developed by LHCb, may be used. However, such a model introduces systematic uncertainties due to modelling.

In this analysis, a model-independent approach is chosen, in which strong phases are determined at a charm factory, BESIII. Here, quantum correlated $D^0\bar{D^0}$ pairs are produced at the $\psi(3770)$ resonance. The amplitude-averaged strong phases are measured in bins of the $D\to K^+K^-\pi^+\pi^-$ phase space. The choice of binning scheme may enhace the sensitivity to $\gamma$. However, a poor choice of binning scheme may only decrease the statistical sensitivity, but not bias the result. With a model-independent approach, one therefore eliminates the systematic uncertainty due to modelling.

%%%%%%%%%%%%%%%%%%%%%%%%%%%%%%%%%%%%%%%%%%%%%%%%%%%%%%%%%%%%%%%
\section{LHCb detector}
\noindent Briefly describe the VELO and RICH

\begin{figure}[H] 
  \centering
  \begin{subfigure}{0.5\textwidth}
    \centering
    \includegraphics[width=1\textwidth]{example-image-a}
    \caption{LHCb detector overview}
  \end{subfigure}%
  \begin{subfigure}{0.5\textwidth}
    \centering
    \includegraphics[width=1\textwidth]{example-image-a}
    \caption{LHCb detector overview}
  \end{subfigure}
  \caption{}
  \label{fig_lhcb_detector}
\end{figure}

%%%%%%%%%%%%%%%%%%%%%%%%%%%%%%%%%%%%%%%%%%%%%%%%%%%%%%%%%%%%%%%
\section{Binning scheme}
\noindent Describe the binning scheme developed and toy studies for testing it

%%%%%%%%%%%%%%%%%%%%%%%%%%%%%%%%%%%%%%%%%%%%%%%%%%%%%%%%%%%%%%%
\section{\texorpdfstring{$B^\pm$}{B} candidate selection}
\subsection{Signal candidate requirements}
\noindent Explain how signal events are selected

\begin{table}[H]
  \centering
  \caption{Requirements}
  \label{data}
  \begin{tabular}{cc} 
    \toprule
    Letter & Numerical value \\
    \midrule
    $c$ & $\SI{299792458}{\meter\per\second}$ \\
    $G$ & $\SI{6.67384e-11}{\newton\meter\squared\per\kilogram\squared}$ \\
    $\hbar$ & $\SI{1.05457e-34}{\joule\second}$ \\
    $k_B$ & $\SI{1.38065e-23}{\joule\per\kelvin}$ \\
    $e$ & $\SI{1.60218e-19}{\coulomb}$ \\
    \bottomrule
  \end{tabular}
\end{table}

\subsection{Background from \texorpdfstring{$D^0\to K^-\pi^+\pi^-\pi^+$}{D->Kpipipi}}
\noindent Show studies of $K3\pi$ contamination

\subsection{Charmless backgrounds}
\noindent Show how flight significance cut removes $B\to KKK\pi\pi$ and mention that $B\to KK\pi\pi\pi$ is insignificant

%%%%%%%%%%%%%%%%%%%%%%%%%%%%%%%%%%%%%%%%%%%%%%%%%%%%%%%%%%%%%%%
\section{Fit to extract CP observables}
\subsection{Global fit and invariant mass spectra}
\noindent State the fit procedure for global fit and show results of global fit for Run $2$, including yields

\subsection{Binned CP fit and CP observables}
\noindent Explain the binned CP fit to extract CP observables

\subsection{Validation of fit procedure with toy studies}

%%%%%%%%%%%%%%%%%%%%%%%%%%%%%%%%%%%%%%%%%%%%%%%%%%%%%%%%%%%%%%%
\section{External strong phase input from BESIII}
\noindent Describe how to extract strong phases at a charm factory and show some initial plots of single tag yields and double tag yields

%%%%%%%%%%%%%%%%%%%%%%%%%%%%%%%%%%%%%%%%%%%%%%%%%%%%%%%%%%%%%%%
\section{Discussion of future work}
\noindent Discuss the plan further


%%%%%%%%%%%%%%%%%%%%%%%%%%%%%%%%%%%%%%%%%%%%%%%%%%%%%%%%%%%%%%%                                                                          
\bibliography{references}
\bibliographystyle{unsrt}

%%%%%%%%%%%%%%%%%%%%%%%%%%%%%%%%%%%%%%%%%%%%%%%%%%%%%%%%%%%%%%%
\newpage
\section{DPhil thesis plan}
\noindent Discuss DPhil thesis plan with Guy first!

\end{document}