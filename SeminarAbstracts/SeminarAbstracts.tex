%%%%%%%%%%%%%%%%%%%%%%%%%%%%%%%%%%%%%%%%%%%%%%%%%%%%%%%%%%%%%%%%%
\documentclass[12pt, a4paper, notitlepage, onecolumn]{article}
\usepackage[UKenglish]{babel}                   % UK style
\usepackage[utf8]{inputenc}
\usepackage[margin=1in]{geometry}               % Margin size
\usepackage{hyperref}                           % Coloured hyperlinks
  \hypersetup{colorlinks = true}
\usepackage{lmodern}                            % Modern fonts
\usepackage{graphicx}                           % For figures
\usepackage[percent]{overpic}                   % For figures with text overlay
\usepackage{amsmath,amssymb}                    % Mathematical symbols
\usepackage{mathtools}
\usepackage{siunitx}                            % SI-units
%\sisetup{exponent-product = \cdot}             % Dot product instead of cross product
\sisetup{separate-uncertainty = true}           % Plus-minus uncertainty
\usepackage{physics}                            % Elegant equations in physics
\usepackage{booktabs}                           % Nice lines, for instance in tables
\usepackage[font=small,labelfont=bf]{caption}% Caption
\usepackage{float}                              % Table do not move with [H].
\usepackage{subcaption}                         % For subfigures
\usepackage[en-GB]{datetime2}                   % UK date format
\usepackage{listings}                           %Source code
\usepackage{feynmp}                             % Feynman diagrams
\DeclareGraphicsRule{*}{mps}{*}{}               % Include Feynman diagrams
\usepackage{scalerel}
\newcommand{\mylbrace}[2]{\vspace{#2pt}\hspace{6pt}\scaleleftright[\dimexpr5pt+#1\dimexpr0.06pt]{\lbrace}{\rule[\dimexpr2pt-#1\dimexpr0.5pt]{-4pt}{#1pt}}{.}}
\newcommand{\myrbrace}[2]{\vspace{#2pt}\scaleleftright[\dimexpr5pt+#1\dimexpr0.06pt]{.}{\rule[\dimexpr2pt-#1\dimexpr0.5pt]{-4pt}{#1pt}}{\rbrace}\hspace{6pt}}
\usepackage{natbib}                      % Set line spacing in references
\setlength{\bibsep}{1.0pt}

%%%%%%%%%%%%%%%%%%%%%%%%%%%%%%%%%%%%%%%%%%%%%%%%%%%%%%%%%%%%%%%
\title{Oxford seminar abstracts 2020-2021}
\author{Martin Duy Tat}
\date{\today}
%\numberwithin{equation}{section}
%%%%%%%%%%%%%%%%%%%%%%%%%%%%%%%%%%%%%%%%%%%%%%%%%%%%%%%%%%%%%%%
\begin{document}
\maketitle
%%%%%%%%%%%%%%%%%%%%%%%%%%%%%%%%%%%%%%%%%%%%%%%%%%%%%%%%%%%%%%%
\section{Higgs Physics: The Next Generation}
\noindent The first decade of LHC physics has seen the discovery of the Higgs boson and the observations of its interactions with vector bosons and the third generation of fermions.  Recent evidence for Higgs interactions with muons moves us into the next generation of Higgs measurements, which will require a new generation of particle colliders.  I discuss the prospects for Higgs measurements at the LHC and future colliders, and their impact on our understanding of the sources of fermion mass, the matter-antimatter asymmetry in the universe, and dark matter.

\section{Searching for Higgs Boson Decays to Light Scalars from Extended Higgs Sectors with the ATLAS Detector}
\noindent As the only non-composite scalar in the Standard Model, associated with an all-permeating, ever-present field thought to generate the mass of all other particles, the study of the Higgs boson represents a historic opportunity to probe reality on the most fundamental level. At the current level of precision all of the measured properties of the Higgs boson are found to be consistent with their SM predictions, and no additional Higgs boson has been observed to date. However, extended Higgs sectors are well-motivated and provide a rich phenomenology of additional scalars. While these new scalars could exist at higher energy scales, they could also exist at or below the electroweak scale, undiscovered by previous experiments, if their most significant coupling is to the Higgs boson. Given the minute natural decay width of the Higgs boson, even small additional couplings to these resonances would lead to final states with substantial, and thus possibly detectable, branching fractions. This talk will introduce two popular extensions of the SM Higgs sector, the Two Higgs Doublet Model and Two Higgs Doublet Model with an additional singlet, before focusing on the searches for decays of the Higgs boson to final states including such light scalars using the ATLAS detector. Emphasis will be placed on a recently published novel search for Higgs boson decays to a Z boson and a hadronically decaying resonance. This final state probes previously inaccessible parts of the 2HDM(+S) parameter space, and is enabled by track-based substructure techniques and a dual-stage neural network.

\section{Belle II – first results from a new flavour physics experiment}
\noindent The seminar will report on the Belle II experiment, a new detector that has recently started taking data at the SuperKEK electron-positron collider after ten years of preparation, construction, and commissioning. We will briefly review the motivation, the design requirements and choices made, and report on the status of the detector. After discussing some early results we will review the physics program of the experiment and present the outlook.

\section{The XENON1T excess, interpretations and implications}
\noindent The XENON1T experiment aims at detecting the specific signatures of WIMP dark matter. The latest analysis showed, however, an unexpected excess of events with low recoil energy. Different potential explanations will be presented including new physics which would lead to the observed excess. Models of new physics have also implications for other experiments which will also be addressed.

\section{Searching for BSM neutrino physics with high-pressure gas TPCs}
\noindent Two decades ago neutrinos provided the first evidence of physics beyond the Standard Model (BSM). In the coming two, they could also be the key to answer some of the outstanding questions in fundamental physics, such as the origin of mass and flavour, or the cosmic asymmetry between matter and antimatter. Many new neutrino experiments with improved sensitivity are being planned. The time projection chamber (TPC) — 50 years after its invention, and already a staple in collider detectors — will have a central role in several of them. 
\\ \\
In this seminar I will talk about two of these new experiment: NEXT and DUNE. The Neutrino Experiment with a Xenon TPC (NEXT) will search for neutrinoless double beta decay using a high-pressure xenon gas TPC. The Deep Underground Neutrino Experiment (DUNE) is a long-baseline neutrino oscillation experiment at Fermilab with primary goals of resolving the neutrino mass hierarchy and measuring the charge-parity violating phase, the indicator of a possible explanation for our matter-dominated universe.

\section{The electron-ion collider: A collider to unravel the mysteries of visible matter}
\noindent Understanding the properties of nuclear matter and its emergence through the underlying partonic structure and dynamics of quarks and gluons requires a new experimental facility in hadronic physics known as the Electron-Ion Collider (EIC). The EIC will address some of the most profound questions concerning the emergence of nuclear properties by precisely imaging gluons and quarks inside protons and nuclei such as their distributions in space and momentum, their role in building the nucleon spin and the properties of gluons in nuclei at high energies. In January 2020 the EIC received CD-0 and Brookhaven National Laboratory was selected as site. This presentation will highlight the capabilities of an EIC, discuss the accelerator design and the concepts for the experimental equipment and give the status of the EIC project.

\section{Hunting for long-lived particles with ATLAS}
\noindent Particles with unusually large lifetimes (called long-lived particles, or LLPs) are predicted in several new physics scenarios, and are a particularly interesting area of research, with connections to dark matter, that could lead to a discovery at the LHC. LLPs can generate unconventional detector signatures that evade the constraints from traditional searches and require dedicated reconstruction techniques to be efficiently detected.
\\ \\
This talk will present recent results from searches for LLPs using 139/fb of pp collision data collected with the ATLAS experiment in Run 2 at the LHC. Several experimental signatures and dedicated techniques are employed, and the results of the searches interpreted as constraints on a variety of BSM models, including the first constraint from ATLAS on long-lived slepton production.

\section{Physics at a Future Energy Frontier Electron-Hadron Collider}
\noindent The Large Hadron electron Collider (LHeC) is a proposed next generation electron-hadron (p,A) machine, for extending the energy frontier in DIS into the TeV regime, beyond HERA. So far, the LHC has not found new symmetries or particles beyond the Standard Model. A strategic query arises as to how we may extend current, incomplete knowledge with a diverse programme. Near-term possible steps for energy frontier physics to progress, prior to a next generation hadron collider, are to build a new electron-positron collider to precisely study the Higgs boson and to realise an electron-hadron collider for sustained discovery and precision with the LHC. Thus talk summarises the status and physics prospects for the LHeC, as summarised in a recent 400-page LHeC paper (arXiv:2007.14491), as well as the FCC-eh, a further future option, proposed to run together with the 100 TeV Future Circular Collider (FCC).

\section{Angular analyses of $B\to K^*\mu^+\mu^-$ decays at LHCb}
\noindent In the past few years there has been increasing interest in $b\to l^+l^-$processes, due to the emergence of several intriguing tensions between measured observables and SM predictions. Of particular interest is the study of angular distributions of such decays, where measurements of angular observables, which carry reduced theory uncertainties, can offer detailed insight on the nature of potential new physics models. This seminar will give an overview of angular analyses of $B\to K^*\mu^+\mu^-$ decays at LHCb, the results of which display some tension with Standard Model predictions. There will also be a brief discussion on prospects for the angular analysis of $B^0\to K^{*0}\mu^+\mu^-$ decays with the full Run 2 dataset.

\section{Search for Light Dark Matter using a Primary Electron Beam}
\noindent The constituents of dark matter are still unknown, and the viable possibilities span a very large mass range. Specific scenarios for the origin of dark matter sharpen the focus on a narrower range of masses: the natural scenario where dark matter originates from thermal contact with familiar matter in the early Universe requires the DM mass to lie within about an MeV to 100 TeV. Considerable experimental attention has been given to exploring Weakly Interacting Massive Particles in the upper end of this range (few GeV – ~TeV), while the region ~MeV to ~GeV is largely unexplored. Most of the stable constituents of known matter have masses in this lower range, tantalizing hints for physics beyond the Standard Model have been found here, and a thermal origin for dark matter works in a simple and predictive manner in this mass range as well. It is therefore a priority to explore. If there is an interaction between light DM and ordinary matter, as there must be in the case of a thermal origin, then there necessarily is a production mechanism in accelerator-based experiments. The most sensitive way, (if the interaction is not electron-phobic) to search for this production is to use a primary electron beam to produce DM in fixed-target collisions. The Light Dark Matter eXperiment (LDMX) is a planned electron-beam fixed-target missing-momentum experiment that has unique sensitivity to light DM in the sub-GeV range. The experiment requires a primary electron beam which is being prepared at the LCLS-II at SLAC. A possible future high energy version proposed for CERN, will also be described.

\section{MINERvA Measurements of Neutrino Interactions in the GeV Regime}
\noindent MINERvA, or Main INjector ExpeRiment for v-A, at Fermilab is an experiment dedicated to the study of neutrino-nucleus interactions in the GeV regime. Its goal is to illustrate the interplay between hadronic and nuclear physics and measure intranuclear dynamics that are crucial for the present and future neutrino oscillation measurements. As the analysis of the Low-Energy data—the  m / m  beam flux peaks at about 3 GeV with most of the rate between 1-6 GeV—has come to a conclusion, nuclear effects are shown to be a complex phenomenon which challenges many of the popular theoretical descriptions. Recently, MINERvA has completed its physics run with the Medium-Energy (flux peak at 6 GeV) beam. The experiment received a total of $12\times 10^{20}$ protons on target in both neutrino and antineutrino mode running, which allow for a new level of statistical precision in neutrino interaction measurements, both in comparisons of interaction channels on a range of nuclei and in expansion to kinematic phase space that has not been accessible in previous data sets. In this seminar, I will overview MINERvA’s Low-Energy results and discuss the Medium-Energy analysis perspective.

\section{Search for time-dependent CP violation in charm decays at LHCb}
\noindent Two years after the first observation of CP violation in the decay of charm mesons by the LHCb experiment, the compatibility of the observation with the Standard Model is still debated. Providing complementary results in time-dependent measurements and in different decay channels is a crucial step towards clarifying whether we are facing new interactions or an enhancement of nonperturbative QCD interactions beyond expectations. This seminar will present a new search for time-dependent CP violation in $D^0\to K^+K^-$ and $D^0\to\pi^+\pi^-$ decays using the LHCb Run 2 data sample. With a precision of $1.5\times 10^{–4}$, it represents the most precise measurement of CP violation performed at the LHC to date, and allows to tighten the bounds on the size of dispersive CP violation in $D^0$ mixing by $35\%$.

\section{Robust test statistics for data sets with missing correlation information}
\noindent Not all experiments publish their results with a description of the correlations between the data points.  This makes it difficult to do hypothesis tests or model fits with that data, since just assuming no correlation can lead to an over- or underestimation of the resulting uncertainties.

I will present new robust test statistics that can be used with data sets with missing correlation information.  They are exact in the case of no correlation and either guaranteed to be conservative -- i.e. the uncertainty is never underestimated -- in the presence of correlations, or they are also exact in the degenerate case of perfect correlation between the data points.

\section{Radio Detection of Astrophysical Neutrinos}
\noindent Neutrino above PeV energies will point us towards the sources of ultra-high energy cosmic rays. However, optical detectors such as IceCube are too small for these low neutrino fluxes. I will discuss radio detection of neutrinos and how we plan to detect neutrinos from 30 PeV to 100 EeV, as part of the Radio Neutrino Observatory Greenland and IceCube-Gen2.

\section{First Results from the Fermilab Muon (g-2) Experiment}
\noindent The Fermilab Muon (g-2) experiment has been taking data since 2018 and is making the world’s most precise measurement at a particle accelerator. In this talk, I will describe the recent analysis of the 2018 dataset that has confirmed the previous measurement of (g-2) made by the Brookhaven, E821, experiment almost 20 years ago. This measurement increases the tension with respect to the Standard Model prediction and adds to the list of flavour anomalies presently circulating in particle physics.

\section{About the universality (or not) of loop induced beauty decays}
\noindent The coupling of the electroweak gauge bosons of the Standard Model to leptons is lepton flavour universal. Extensions of the Standard model do not necessarily have this property. Rare decays of heavy flavour are heavily suppressed in the Standard Model and new particles can give sizeable contributions to these processes, thus their precise study allows for sensitive tests of lepton flavour universality. Of particular interest are rare $b\to sll$ decays that are readily accessible at the LHCb experiment. Recent results from LHCb on lepton flavour universality in rare $b\to sll$ decays are discussed.

\section{Recent results on leptonic and semileptonic $D_{s}^{+}$ decays}
\noindent Leptonic and semileptonic decays of charmed hadrons provide unique tests of our theories of electroweak and strong interactions. In this talk, I’ll present two recent results from the BESIII experiment relating to (semi)leptonic decays of $D_{s}^{+}$ mesons: an analysis of $D_{s}^{+}$ decaying to pure leptonic final states, and an analysis of inclusive semielectronic $D_{s}^{+}$ decays. The first analysis provides the world’s most precise single measurements of both the pure tauonic and muonic branching fractions of the $D_{s}^{+}$. These, in turn, provide precise determinations of the $D_{s}^{+}$ decay constant, the $|V_{cs}|$ CKM element, and a charm-sector lepton-universality test. The second analysis measures the inclusive semielectronic $D_{s}^{+}$ branching fraction, sets an upper limit on the unobserved $D_{s}^{+}$ semielectronic branching fraction, and provides a testing ground for techniques of extracting CKM elements from inclusive semileptonic meson decays.

\section{Model-independent $D^0\to 4h$ \& LHCb PID Calibration}
\noindent Recent years have seen a renaissance of charm physics fueled by a range of new results. $D$ meson mixing is now well-established, and CP violation in the charm system was observed in 2019. Experimental studies of charm mixing are sensitive to two parameters, $x = \Delta M/\Gamma$ and $y = \Delta \Gamma/2\Gamma$. Measurements show $y\approx 1\%$, clearly establishing a non-vanishing neutral charm decay width difference, and it is assumed x is of similar magnitude. However, the measurements of x are so far still consistent with zero.

 Particle identification (PID) is a crucial ingredient of the LHCb physics programme. It combines information from many sub-detector systems to distinguish final state particles. LHCb analyses currently use a PID calibration framework that limits the dimensionality of the PID efficiency binning. As the datasets of Run 3 will be much larger than before, we must control the systematic uncertainties to a better degree. This will require a higher-dimensional binning than the current approach allows.

\section{The FASER Experiment at the CERN LHC: Looking Forward to New Physics}
\noindent The Forward Search Experiment (FASER) is the newest experiment at the LHC, approved in 2019 and recently installed into the CERN LHC complex. It is a small and inexpensive experiment placed 500 meters downstream of the ATLAS interaction point. FASER is designed to capture decays of exotic particles, produced in the very forward region, out of the ATLAS detector acceptance. In addition, FASERnu, a FASER sub-detector, is designed to detect collider neutrinos for the first time and study their properties. This seminar will present the physics prospects, the detector design, and the construction and installation progress of FASER.

\section{3D Sensors Status and Perspectives}
\noindent 3D sensors with electrodes penetrating the substrate's bulk have been a game changing technology in radiation hard particle tracking. After 25 years from the original proposal, their industrialization and inclusion in the ATLAS IBL, they are being explored for features beyond radiation hardness like fast timing and spatial precision. The seminar will present the historical evolution of 3D technology, its technical characteristics and how the technology could respond to future challenges in particle tracking.

\section{Precision Higgs Physics at a future e+e- Collider}
\noindent The next goal for high-energy accelerator experiments is the precision study of the Higgs boson. This can be accomplished by an electron-positron collider covering the energy region 250 GeV - 1 TeV. There is now an opportunity to construct an accelerator of this type, an e+e- Higgs Factory. In this talk, I will review (1) why the study of the Higgs boson is so important, and how it can open a new window on physics beyond the Standard Model, (2) why e+e- colliders give important advantages for the study of the Higgs boson, (3) how Higgs boson measurements fit into a more general program of Standard Model precision tests, and how this viewpoint enhances the power of the Higgs boson measurements, (4) the prospects for actually realizing this program of experiments at the International Linear Collider in Japan.

\section{Unexpected discoveries and faster simulations through unsupervised learning}
\noindent Machine learning already achieves excellent results for traditional object reconstruction and classification tasks in particle physics. It typically exceeds classical benchmarks where the learning objective is relatively straightforward to define. Here, we review the potential of weakly supervised and unsupervised learning techniques - i.e., algorithms that can learn from data more directly.  We first discuss the quest for building model-independent searches for new physics (a.k.a, anomaly detection). We then consider how generative machine learning models can amplify and speed up traditional simulations.

\end{document}